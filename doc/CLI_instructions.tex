The contents of a configuration file are shown below:
\newpage
\begin{itemize}

\begin{ttfamily}
\item[]
\# \\
\#	This is an example command line interface (cli) configuration file. \\
\#	 \\
\#	This particular file illustrates the necessary options settings to \\
\#	do a weather forecast model initialization run with diurnal winds. \\
\#	The weather model is downloaded via the Internet.  The mesh is set \\
\#	to a specified resolution of 250 meters. \\
\# \\
num\_threads				=	12 \\
elevation\_file				=	C:/XXXX/missoula\_valley.tif \\
initialization\_method			=	wxModelInitialization \\
time\_zone					=	America/Denver \\
wx\_model\_type				=	NCEP-NAM-12km-SURFACE\\ 
forecast\_duration			=	100 \\
output\_wind\_height			=	20.0 \\
units\_output\_wind\_height		=	ft \\
vegetation				=	trees \\
diurnal\_winds				=	true\\ 
mesh\_resolution				=	250.0 \\
units\_mesh\_resolution			=	m \\
write\_goog\_output			=	true \\
write\_shapefile\_output			=	true\\ 
write\_ascii\_output			=	true \\
write\_farsite\_atm			=	true \\
write\_wx\_model\_goog\_output		=	true\\ 
write\_wx\_model\_shapefile\_output	=	true \\
write\_wx\_model\_ascii\_output		=	true\\
\end{ttfamily}

\end{itemize}

To run this particular configuration file, you would just type:

\begin{itemize}
\item[] \texttt{WindNinja\_cli C:/XXXX/cli\_wxModelInitialization\_diurnal.cfg}
\end{itemize}

where \say{\texttt{XXXX}} represents the rest of the path to the file.

\section*{Starting a run by specifying options from both the terminal and a configuration file}

A very useful feature of the WindNinja cli is that you can specify options from both the terminal and a configuration file at the same time.  One way to use this feature would be to put the more \say{general} options in a configuration file, but then specify  other more specific options for the run via the terminal.  If the same option is specified in both the terminal and the configuration file, the terminal value is used.

As an example of this, you could use the configuration file shown above but “override” the elevation file, vegetation, and number of threads options by typing this:

\begin{itemize}
\item[] \texttt{WindNinja\_cli C:/XXXX/cli\_wxModelInitialization\_diurnal.cfg  --elevation\_file C:/XXXX/canyon\_fire.asc --vegetation grass --num\_threads 4}
\end{itemize}

%This isn't included anymore with the default Windows Installation
%commenting out for now
%\section*{Response files}
%Some operating systems (such as older UNIX systems and certain Windows variants)[\href{https://gcc.gnu.org/wiki/Response_Files}{1}] have very low limits of the command line length. One common way to work around those limitations is using response files (instead of configuration files). A response file is just a configuration file which uses the same syntax as the command line (rather than the configuration file syntax described above). If the command line specifies a name of response file to use, it's loaded and parsed in addition to the command line.  We recommend using a configuration file rather than a response file simply because the syntax is more readable and comments are allowed.  The response file method is provided for rare situations where using it might be necessary.
%
%An example response file is located in the installation's “example-files” directory called:
%\begin{itemize}
%\item[] \texttt{cli\_domainAverage\_diurnal.rsp}
%\end{itemize}
%
%To run this response file, you would just type:
%\begin{itemize}
%\item[] \texttt{WindNinja\_cli @C:/XXXX/cli\_wxModelInitialization\_diurnal.rsp}
%\end{itemize}
%
%where \say{\texttt{XXXX}} represents the rest of the path to the file.
%Notice the \say{\texttt{@}} character preceding the response file name, which identifies it as a response file.

\section*{Tutorial Cases}

The WindNinja tutorials provide a practical way to learn how to effectively use the WindNinja Command Line Interface. Cases are provided for all initialization methods. These tutorials are easy to run and provide instructions and explanations for various  use cases.

\subsection*{   Locating the Tutorial Directory}
The tutorial directory is located within the WindNinja data directory. Refer to the README file within the tutorial directory for directions on how to set up your own tutorial directory, and how to run the cases.  


\section*{Types of Runs and Required Input Options}

This section outlines the 3 different initialization types for WindNinja runs, along with the most basic required inputs for each type of run. These options can be set in a .cfg file, or specified in the terminal as detiled above. Additional options can be viewed by typing: 
\begin{itemize}
\item[]\texttt{WindNinja\_cli} 
\end{itemize}



\subsection*{Required inputs for ALL runs}

\textbf{Required Inputs:}
\begin{itemize}
    \item \texttt{elevation\_file}
    \item \texttt{initialization\_method}
    \item \texttt{output\_wind\_height}
    \item \texttt{units\_output\_wind\_height}
    \item \texttt{vegetation}
    \item Either \texttt{mesh\_resolution} or \texttt{mesh\_choice}
    \item \texttt{units\_mesh\_resolution} - Required if \texttt{mesh\_resolution} is used
\end{itemize}

\subsection*{Point Initialization}
If the fetch type option is included in the Point Initialization run, additional inputs are required:

\begin{itemize}
    \item If \texttt{fetch\_type} = `Stid' (station ID):
        \begin{itemize}
            \item \texttt{wx\_station\_filename}
        \end{itemize}
                       
    \item If \texttt{fetch\_type} = `bbox' (bounding box):
        \begin{itemize}
            \item \texttt{fetch\_station} - Should be set to true
        \end{itemize}
\end{itemize}




\subsection*{Domain Average Initialization}

\textbf{Required Inputs:}
\begin{itemize}
    \item \texttt{input\_speed}
    \item \texttt{input\_units} 
    \item \texttt{input\_direction}
    \item \texttt{input\_wind\_height}
    \item \texttt{units\_input\_wind\_height} 


\subsection*{Weather Model Initialization}

\textbf{Required Inputs:}
\begin{itemize}
    \item \texttt{wx\_model\_type} - Type of the weather model (e.g., `NCEP-NAM-12km-SURFACE`)
    \item \texttt{forecast\_duration} 


\end{document}
